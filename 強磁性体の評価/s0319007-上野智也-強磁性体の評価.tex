\documentclass[dvipdfmx]{jsarticle}
\usepackage[dvipdfmx]{graphicx}
\graphicspath{{ピクチャ/}}
\usepackage{otf}
\usepackage[dvipdfmx]{graphicx}
\usepackage{amsmath,amssymb}
\usepackage{url}
\usepackage{bm}
\usepackage{wrapfig}
\usepackage{tcolorbox}
\tcbuselibrary{skins}
\begin{document}

\title{\huge II-8強磁性体の評価}
\date{}
\maketitle

\begin{flushright}
\vspace{11cm}\Large
所属:物理・材料理工学科\\数理・物理コース\\
学籍番号:s0319007\\
氏名:上野智也\\
共同実験者:上野健斗\\片野峻太郎\\川瀬悠太\\
実験実施日:2021/12/2\\
レポート提出日:2021/12/13
\end{flushright}

\thispagestyle{empty}

\newpage

\section{目的}

強磁性体の温度依存性を測定し、磁性に関する物理学的知識を習得する。

\section{原理}

強磁性体は大きな比磁化率($\bar \chi=10^3~10^6$)を持ち、強い磁界の下で飽和磁化$I_s$を示す。これは、原子の持つ磁気モーメントが相互作用の下に平行に配列することによって自発磁化を持つためである。飽和磁化は温度上昇とともに低下し、キュリー点$\Theta$で消失して0になる。これ以上の温度では磁化率の逆数$1/\chi$は直線的に増加する。(図1)

磁性の担い手は電子の軌道運動及びスピン、原子核の運動及びスピンが考えられる。電荷を持った電子の軌道運動は感情電流に相当する一方この軌道運動は同時に角運動量をもつため、

\begin{equation}
\displaystyle \bm{M}_0=-\frac{\mu_0e}{2m}\bm{P}_0
\end{equation}

の磁気モーメントが発生する。ここで、$eは電子の電荷、mは電子の質量、\mu_0は真空での透磁率。\bm{P_0}(=\bm{l}\hbar)は$軌道角運動量であり、軌道量子数$l(=0,\pm1,\pm2,...)$で指定されるとびとびの値しかない。また、電子のスピン角運動量も

\begin{equation}
\displaystyle \bm{M}_s=-\frac{\mu_0e}{2m}\bm{P}_s
\end{equation}

のスピン磁気モーメントを作る。ここで、$\bm{P_s=s}\hbar はスピン角運動量であり、スピン量子数は\pm1/2の値をもつ。よって磁気モーメントMと角運動量Pの関係は一般的に、$

\begin{equation}
\displaystyle \bm{M}=-g\frac{\mu_0e}{2m}\bm{P}
\end{equation}

の関係がある。$g$はジャイロ磁気定数で、軌道運動に対して$g=1、スピンに対してg=2となる。$

原子核の場合、固体中での核の運動は格子点に束縛された振動で軌道運動ではないので磁気モーメントを持たない。


\begin{wrapfigure}[5]{r}{15zw}
 \vspace*{-\intextsep}
 \includegraphics[width=15zw]{1.jpg}
 \caption{強磁性体の飽和磁化の温度変化}
\end{wrapfigure}

一方、核が持つスピンは磁気モーメントを発生するが、陽子の質量は電子に比べて2000倍も大きいので、物質の磁化にほとんど関与しない。したがって、原子の磁気モーメントは電子の軌道運動量とスピンによるものであるが実際の強磁性体の磁性は電子スピンが担っていることがジャイロ磁気係数などの測定からわかっている。

\newpage

磁性の本性は、スピン磁気モーメントを単位としたその協力現象として解釈でき、常磁性の磁化を統計的に扱ったランジュバンの常磁性理論が出発点となる。

磁気モーメント$M$を持つ原子が単位体積中に$N個あり、温度T$ [K]で磁界$Hを加えた時、ボルツマン統計を用いると結果としてH方向の磁化の強さは、$

\begin{equation}
\displaystyle I=NM(\coth \alpha -\frac{1}{\alpha})
\end{equation}

で表される。ここで、$\alpha = MH/kTであり、()内の関数はランジュバン関数と呼ばれる。H \to \infty のとき、I \to MNに近づくことを表しており、充分強い磁界中でスピンが平行にそろうことを意味している。$

$\alpha<<1の場合は$、(4)式の展開第一項のみとると、

\begin{equation}
\displaystyle I\simeq NM\coth \alpha \simeq \frac{NM^2H}{3kT}
\end{equation}

となるから、磁化率は

\begin{equation}
\displaystyle \chi =\frac{1}{H}=\frac{NM^2}{3k} \frac{1}{T}
\end{equation}

となり常磁性磁化率は絶対温度$T$に逆比例することが分かる。これをキュリーの法則という。

ランジュバンの理論では磁気モーメントはあらゆる方向をとりえると仮定したが、実際には空間量子化されていることを考慮しなくてはならない。この時、平均磁化$I$は

\begin{equation}
\displaystyle I=NM[\frac{2J+1}{2J}\coth(\frac{2J+1}{2J}\alpha)-\frac{1}{2J}\coth(\frac{1}{2J}\alpha)]
\end{equation}

となる。ここで、[]の中はブリルアン関数$B_J(\alpha)$で、$\alpha=MH/kT$である。$Jは合成全角運動量であり、合成軌道角運動量と合成スピン角運動量からなる。$

$\alpha<<1の場合、B_J(\alpha)は展開出来て、第一項までとると、磁化は$

\begin{equation}
\displaystyle I\simeq\frac{(J+1)NM^2H}{3JkT}
\end{equation}

となるから、磁化率は以下の式で与えられる。

\begin{equation}
\displaystyle \chi = \frac{I}{H}=\frac{(J+1)NM^2}{3Jk}\frac{1}{T}
\end{equation}

ランジュバンの理論では、スピンが平行に揃うには$10^{10}$A/m(~10$^8$Oe)もの強い磁界が必要とされるが、実際の強磁性体では1~10$^4$A/m程度でスピンは揃い、磁化は飽和する。これは、磁気モーメント間につよい相互作用が働いてたがいに平行に並べているからで、ワイスにより分子場磁界という概念として導入された。

ワイスによると、磁化された領域の中で一つのスピンが周囲のスピン集団から受ける磁界は、磁化の強さ$I$に比例した分子場磁界$wI$であり、したがって個々のスピンは外部磁界$H$と分子磁界$wI$の両者の作用を受ける。この条件下での平均の磁化は量子化されたボルツマン統計にしたがうと以下の式のようになる。

\begin{equation}
\displaystyle I=NMB_J(\alpha) , \alpha =\frac{M(H+wI)}{kT}
\end{equation}
\newpage

$\alpha$は書き換えると、

\begin{equation}
\displaystyle I=\frac{kT}{Mw}\alpha-\frac{H}{w}
\end{equation}

と表される。$I$についての2つの式(10),(11)の連立方程式を解けば、$T=0のとき、I=NMであり、温度が上昇するとT=\Theta 以上でIが常にゼロになる。この\Theta をキュリー点という。$

\begin{wrapfigure}[15]{r}{15zw}
 \vspace*{-\intextsep}
 \includegraphics[width=15zw]{2.jpg}
 \caption{ワイス理論による自発磁化の温度変化}
\end{wrapfigure}

縦軸を磁化の強さ$I/I_0、横軸を温度T/\Theta で表すと自発磁化の温度変化は様々なJの値に対して図$2のようになる。

$\Theta は$

\begin{equation}
\displaystyle \Theta =\frac{(J+1)NM^2w}{3Jk}
\end{equation}

となり、磁化率$\chi$は

\begin{equation}
\displaystyle \chi = \frac{(J+1)NM^2}{3Jk(T-\Theta)}=\frac{N\mu^2_{eff}}{3k(T-\Theta)}
\end{equation}

で与えられる。磁化率$\chi$は$T=\Theta で発散してキュリー点からずれT-\Theta に反比例する。これをキュリー・ワイス則という。$また、式中の$\mu_{eff}$は磁性元素あたりの有効磁気モーメント

\begin{equation}
\displaystyle \mu_{eff}=g_J \mu_B\sqrt{J(J+1)}
\end{equation}

であり、導出には$M=g_J \nu_B J$の関係を用いた。ここで、ボーア磁子を$\mu_B=\mu_0 e \hbar /(2m)$、ランデの$g因子をg_J$とした。有効磁気モーメントはキュリーワイス則から実験的に求められ、フント則に基づいた理論値とよく一致することが知られている。

式(12)から分かるようにキュリー温度$\Theta$と分子場磁界係数$w$は比例関係にある。古典モデルでの$w$の1つの評価はローレンツの局所磁場である。それによると、$w_L=1/(3\mu_0)=2.65\times 10^5$ m/Hとなる。一方、今キュリー点$\Theta$=1063 Kを持つ鉄を例にとって、$M=2.2 \mu_B、N=8.54 \times 10^{28}$ m$^{-3}$とすると、式(12)殻分子磁界系数は$w=3.9\times 10^6$ m/Hとなり、三桁ほど大きくなっている。このはるかに大きい$w$の値は、量子論の立場から磁気モーメントが量子スピンによるものであり、かつパウリの禁制律とクーロン相互作用を組み合わせたスピン同士の相互作用を考慮して初めて説明される。

\newpage

\section{実験手順}

\subsection{測定試料}

ニッケル及びジスプロシウム(共に純度99.9\%)のリング型多結晶試料を用いる。

\subsection{磁界の発生と検出}

図3に測定装置とその回路を示す。試料の周りに巻きつけた励磁コイルに、波形発生器および定電流電源を通して生成した三角波電流を通して磁界を発生させる。コイルの巻き数を$n_1$、電流の強さを$i$ [A]、コイルの長さを$l$ [m]とすると、コイル内部の磁界$H$[A/m]は、アンペールの定理より、

\begin{equation}
\displaystyle H=\frac{n_1i}{l}
\end{equation}

となる。とくに、外径、内径がそれぞれ$d_1,d_2$ [m]のリング試料では、コイル長は$l=\pi(d_1-d_2)/\log (d_1/d_2)$で与えらえれる。励磁コイルに直列につないだ抵抗値1$\Omega$の抵抗間の電圧$V_i$を測定することで、電流の強さ$i$が求まり、よって試料内部の磁界$H$が決定できる。

\begin{figure}[h]
\centering
\includegraphics[width=10cm]{3.jpg}
\caption{測定装置および回路図:(a)低温測定用、(b)高温測定用(試料部周りのみ記載)}
\end{figure}%

\subsection{磁化の強さの測定}

磁化の強さの測定は、試料周りに巻いた検出コイルを用いた電磁誘電法で行う。磁界を試料の円周方向(検出コイル面に垂直)に印加すると、検出コイルの両端には以下の起電力$V_2$が生じる。

\begin{equation}
\displaystyle V_2=-n_2S\frac{dB}{dt}
\end{equation}

ここで、$n_2は検出コイルの巻き数、S$ [m$^2$]は試料の断面積、$B$ [T](テスラT=Wb/m$^2$)は磁束密度である。
\newpage

磁束密度$B$は磁化の強さ$I$ [T]と磁界$H$ [A/m]の間に$B=I+\mu_0H$の関係があるので、

\begin{equation}
\displaystyle V_2=-n_2S(\frac{dI}{dt}+\mu_0\frac{dH}{dt})
\end{equation}

となる。ここで、真空透磁率$\mu_0$は$4\pi \times 10^{-7}$ H/mである。本測定では磁界$H$は時間変化しているので、磁界$H$による誘電起電力も生じる。測定値$V_2$を時間に対して積分することで、ある磁界$H$における磁束密度$B$が求められる。
その値から外部磁界$H$の影響を差し引くことで、磁化の強さ$I$が決定される。ただし、強磁性体の中でも特に軟磁性体の場合は、磁化$I$の時間変化に伴う誘導起電力((17)式の第一項)が圧倒的に大きいので、$B\simeq Iとしても差支えない。$

\subsection{実験手順(ニッケルの磁化の温度変化)}

(1)試料:リング資料(外径12mm、内径10mm、厚さ3mm)にすでに検出コイル(70巻き)、励磁コイル(60巻き)が巻かれている。高温下(室温~400℃)で測定するため、コイルに耐熱被覆銅線が用いられており、また試料部はセラミックでコーティングされている。

(2)制御系パソコン、定電流電源、波形生成器、信号増幅器、フィルターの電源が入っている事を確認し、ニッケル用の磁化測定プログラムを立ち上げる。なお、信号増幅器およびフィルターはS/N比をよくするために用いられており、すでに最適化設定されている。

(3)温度表示器およびスライダックをAC100V電源に接続する。この時スライダックのダイヤルが0V、温度表示器の温度が室温程度であることを確認する。

(4)図3(b)のように、試料および熱電対を炉内のタンマン管に挿入する。

(5)試料部からの配線(励磁コイル(+)(-)、検出コイル(+)(-))を端子台を通して、各機器に接続。

(6)磁化測定プログラムを用いて、室温の$B-H$曲線を測定して最大印加磁場における磁束密度を求める。励磁周波数は1 Hz、最大磁場は8 kA/m程度とする。

(7)スライダックのダイヤルを70V程度に合わせる。

(8)温度表示器の温度(試料温度)が徐々に上昇するので、20~30℃ごとに400℃まで磁化曲線を測定し、測定時の温度と飽和磁化を記録する。260~380℃では5~10℃ごとの測定が望ましい。また、測定時は特に温度を安定させる必要はない。

(9)400℃まで測定が終了したら、スライダックのダイヤルを0Vに戻し、AC100Vのコンセントを抜いておく。

\subsection{実験手順(ジスプロシウムの磁化の温度変化)}

(1)試料:リング試料(外径9mm、内径7mm、厚さ3mm)にすでに検出コイル(100巻き)、励磁コイル(52巻き)が巻かれている。ジスプロシウムは室温では常磁性であり、低温($<$85 K)で強磁性を示す。そのため、図3(a)の液体窒素デュワーを用いて磁化測定を行う。試料および白金抵抗温度計はステンレス製試料支持棒の先端にある銅製試料ホルダ内に取り付けられている。

(2)制御系パソコン、定電流電源、波形生成器、信号増幅器、フィルターの電源が入っていることを確認して、ジスプロシウム用の磁化測定プログラムを立ち上げる。

(3)液体窒素デュワーに、1/3程度の深さまで液体窒素を満たす。

(4)試料部殻の配線(励磁コイル(+)(-)、検出コイル(+)(-)、温度計用定電流源(+)(-)、温度計用電圧計(+)(-))を各機器の接続する。

(5)白金抵抗温度計用の電圧計の電源、および定電流電源(電池式)のスイッチを入れる。電圧計の測定値から白金の抵抗-温度換算表中の抵抗値と比較し、抵抗温度計が正しく動作していることを確認する。

(6)試料支持棒をゆっくり液体窒素デュワー内に挿入する。このとき、支持棒は一気に入れず、まず先端のみ液体窒素につけて、液体窒素の蒸発が収まってから挿入し始める。

(7)試料支持棒の上部を、スタンドのクランプではさんで固定する。

(8)白金抵抗温度計の測定電圧値から温度を求めて、試料部が液体窒素の沸点温度(~77K)になっていることを確認。

(9)磁化測定プログラムを用いて、液体窒素温度の$B-H$曲線を測定して最大印加磁場における磁束密度を求める。励磁周波数は5 Hz、最大磁場は8 kA/m程度。

(10)試料支持棒を少し引き上げて、試料温度を徐々に上げながら、$B-H$曲線を計測する。2~3 Kごとに室温まで測定を行う。

\newpage

\section{課題}

(1)磁化$I$ [T]と温度$T$ [K]の関係のグラフを作成せよ。なお、キュリー点以上の十分高温で測定した磁束密度$B$の実験値を外部磁界による磁束密度成分をみなして、この値を全体の測定値から差し引いて磁化Iを決定すること。

  

・ニッケルの磁化$I$ [T]と温度$T$ [K]の関係のグラフ

  

ニッケルの時、磁化$I$は測定磁束密度$B$ [T]から測定磁界$H$ [A/m]に真空の透磁率$\mu_0=4\pi \times 10^{-7}$ [H/m]をかけたものを引いた値である((18)式)。ニッケルの場合、測定磁界$H$の変化が少ないため平均をとり、それを引くこととして$I$を求めた。以下のグラフはそうして求めた磁化$I$と温度$T$の関係を表すグラフである。グラフ中のオレンジ色の点線と数式は課題(2)で用いる近似直線とそれを表す数式である(ただし、$y=I,x=T$)。

\begin{equation}
\displaystyle I=B-\mu_0 H
\end{equation}

\begin{figure}[h]
\centering
\includegraphics[width=12cm]{niit.jpg}
\caption{ニッケルの磁化$I$ [T]と温度$T$ [K]の関係のグラフ}
\end{figure}%

\newpage

・ジスプロシウムの磁化$I$ [T]と温度$T$ [K]の関係のグラフ

  

ジスプロシウムのときも磁化は(18)式から求められるが、今回は$\mu_0 H=0.009$ [T]として計算して磁化$I$を求める。磁化$I$と温度$T$の関係を以下のグラフに示す。だだし、グラフ中のオレンジ色の点線とその近くの数式は、考察で使用する近似直線とその式である(ただし、$y=I,x=T$)。

\begin{figure}[h]
\centering
\includegraphics[width=12cm]{DYIT.jpg}
\caption{ジスプロシウムの磁化$I$ [T]と温度$T$ [K]の関係のグラフ}
\end{figure}%

(2)ニッケルについて$I/I_0-T/\Theta$ の曲線を作図せよ。このとき、300 Kと0 Kの飽和磁化の比を0.95とする。また、キュリー点$\Theta$は、飽和磁化の温度変化が極大になるところに引いた接戦が温度軸を横切る温度とする。

  

300 Kと0 Kの飽和磁化の比が0.95であるが、今回の実験で得られた値でもっとも300Kに近い値が296Kであったため296Kと0Kの飽和磁化の比が0.95であると仮定して0Kの時の飽和磁化$I_0$を求める。296Kの時の飽和磁化を$I$として下に式に代入して0Kの時の飽和磁化$I_0$を求めた。

\begin{center}
$\displaystyle \frac{I}{I_0}=0.95 すなわち I_0=\frac{I}{0.95}$
\end{center}

キュリー点を求める。図4のグラフで、飽和磁化の温度変化が極大になるところは、623Kと633Kの2点の間であった。この二点をExcelの近似直線の機能を用いて直線で結び、温度軸との交点を求める。直線は図4中のオレンジ色の点線であり、その式は$y=-0.0103+6.5359$となっている。この式で$y=0$とすれば温度軸との交点、すなわちキュリー点$\Theta$が導ける。実際に計算したところ、キュリー点は$\Theta=635$Kであった。

キュリー点以上でニッケルは常磁性となってしまうため、$T/\Theta$の範囲は0~1とする。

以上のことからニッケルについて$I/I_0-T/\Theta$ の曲線を求めると次のページのグラフになる。

\newpage

\begin{figure}[h]
\centering
\includegraphics[width=12cm]{ni2.jpg}
\caption{ニッケルの$I/I_0-T/\Theta のグラフ$}
\end{figure}%

(3)ジスプロシウムについて磁化率$\chi$を計算して、磁化率の逆数$1/\chi$と温度$T$の関係のグラフを作成せよ。なお、磁化率は磁化の測定値$I$ [T]と磁界$H$ [A/m]の間に、以下の関係がある。

\begin{equation}
\displaystyle \chi = \frac{I}{H}
\end{equation}

磁界$H$の平均をとり、測定値$I$を(19)式に代入することで磁化率を求めた。以下にジスプロシウムの磁化率の逆数$1/\chi$と温度$T$の関係のグラフを示す。ただし、グラフ中のオレンジの点線とその近くの数式は考察5.3(2)で用いるものでExcelで出した近似直線とそれを表す式である(ただし、$y=1/\chi,x=T$)。

\begin{figure}[h]
\centering
\includegraphics[width=11.9cm]{dy3.jpg}
\caption{ジスプロシウムの磁化率の逆数$1/\chi$と温度$T$の関係}
\end{figure}%

\newpage
\section{考察}

\subsection{測定試料について}

(1)測定ではリング状試料を用いる。このメリットとデメリットを外部磁場強度の観点から考察せよ。

  

リング状試料と棒状の試料を比較して考えてみる。

  

・メリット

棒状の試料に電流を流すと磁極が発生して磁束が流れ出してしまい、逆向きの磁束が試料の周りに発生してしまい測定への影響が考えられる。リング状の試料ならば磁極が発生せず、磁束が外に漏れださないため、磁束の変化を棒状の試料よりも正確に測定できると考えられる。(図8)

\begin{figure}[h]
\centering
\includegraphics[width=8cm]{ring.jpg}
\caption{リング状の試料と棒状の試料の比較}
\end{figure}%

  

・デメリット

コイルに電流を流し続けるとジュール熱が発生して温度計に誤差を与えてしまうかもしれない。

棒状の試料は作成が比較的容易であるが、リング状の試料は加工が難しい。

\newpage

\subsection{ニッケル}
 
(1)室温の飽和磁化とキュリー点の実験値を文献値と比較して考察

実験値と文献値をまとめると以下のようになる。

\begin{table}[h]
  \centering
    \begin{tabular}{|r||r|r|} \hline
     & 実験値 & 文献値 \\ \hline
室温の飽和磁化$I$[T] & 0.50 & 0.61 \\ \hline
キュリー点$\Theta$[K] & 635 & 631 \\ \hline
    
    \end{tabular}
    \caption{実験値と文献値の比較(ニッケル)}
    \label{tab:r_1}
\end{table}

・考察

室温での飽和磁化の実験値が文献値よりも小さくなった。図4から室温の飽和磁化よりも50℃と80℃の飽和磁化の方が大きいことがわかる。Niの飽和磁化は高温ほど小さくなるため、室温で磁化しきる前に測定してしまい、小さい値が出たのだと考えられる。

キュリー点の温度が文献値よりも高い原因として考えられるのは、測定温度での磁化の大きさに減少しきる前に磁化の大きさを測定してしまったため、本来相転移する温度での磁化の大きさに達せず、より高い温度のときに相転移する磁化の大きさになったからだと考えられる。

実験値をより文献値に近づけるには、キュリー点付近の温度で、今回の実験よりもゆっくりと測定を行うという方法が考えられる。

  

(2)式(11)と(12)から$J=1/2,1,\infty についてのI/I_0-T/\Theta 曲線を求め、課題(2)で作図した実験値と比較せよ。なお、式(11)
はH=0としてよい。どのJの計算値と合うか調べ、その理由を述べよ。$

  

\begin{center}
$\displaystyle I=NMB_J(\alpha)$ (10)

$\displaystyle I=\frac{kT}{Mw}\alpha$ (11)

$\displaystyle \Theta =\frac{(J+1)NM^2w}{3Jk}$ (12)

ただし、$\displaystyle \alpha=\frac{MH}{kT}$
\end{center}

以上の3式から$J=1/2,1,\infty についてのI/I_0-T/\Theta 曲線を求める。$

$B_J(\alpha)$はブリルアン関数であるため以下の式のようになる。

\begin{center}

$\displaystyle B_J(\alpha)=\frac{2J+1}{2J}\coth(\frac{2J+1}{2J}\alpha)-\frac{1}{2J}\coth(\frac{1}{2J}\alpha)$

\end{center}

ここで、

\begin{center}
$\displaystyle \coth x=\frac{e^x+e^{-x}}{e^x-e^{-x}}=\frac{1+e^{-2x}}{1-e^{-2x}}より、x\to \infty のとき\coth x=1$となる。
\end{center}

$したがって、T\to 0のとき、\alpha \to \infty なのでT=0の時の磁化IをI_0とすると$

\begin{center}
$\displaystyle I_0=NM(\frac{2J+1}{2J}-\frac{1}{2J})=NM$となる。
\end{center}

$I_0で$(10),(11)式の両辺を割ると以下のようになった。

\begin{center}
$\displaystyle \frac{I}{I_0}=B_J(\alpha)$ (10')

$\displaystyle \frac{I}{I_0}=\frac{kT}{NM^2w}\alpha$ (11')
\end{center}

ここで、(12)式は以下のように変形できる。

\begin{center}
$\displaystyle \frac{NM^2w}{k}=\frac{3J}{J+1}\Theta$ (12')
\end{center}

(12')式を(11')式に代入すると

\begin{equation}
\displaystyle \frac{I}{I_0}=\frac{J+1}{3J}・\frac{T}{\Theta}\alpha
\end{equation}

(10'),(20)式の交点から$\alpha$が決定できて、$I/I_0$が求められる。

  

(i)$\displaystyle J=\frac{1}{2}のとき$

$\displaystyle J=\frac{1}{2}を(20),(10')式に代入すると以下のようになる。$

\begin{center}

$\displaystyle \frac{I}{I_0}=\frac{1/2+1}{3/2}\frac{T}{\Theta}\alpha=\frac{T}{\Theta}\alpha$

$\displaystyle \frac{I}{I_0}=B_{1/2}(\alpha)=2\coth 2\alpha -\coth \alpha$
\end{center}

上の2式のグラフの交点から$\alpha の値を求めて、式(20)に\alpha を代入してI/I_0-T/\Theta 曲線を求める。$

  

(ii)$J=1$のとき

$J=1$を(20),(10')式に代入すると以下のようになる。

\begin{center}
$\displaystyle \frac{I}{I_0}=\frac{2}{3}\frac{T}{\Theta}\alpha$

$\displaystyle \frac{I}{I_0}=B_1(\alpha)=\frac{3}{2}\coth \frac{3}{2}\alpha - \frac{1}{2}\coth \frac{1}{2} \alpha$
\end{center}

上の2式のグラフの交点から$\alpha の値を求めて、式(20)に\alpha を代入してI/I_0-T/\Theta 曲線を求める。$

  

(iii)$J=\infty$のとき

$J=\infty$を(20),(10')式に代入すると以下のようになる(上から2番目の式はランジュバン関数になっている)。

\begin{center}
$\displaystyle \frac{I}{I_0}=\frac{J+1}{3J}\frac{T}{\Theta}\alpha=\frac{1+\frac{1}{J}}{3}\frac{T}{\Theta}\alpha=\frac{1}{3}\frac{T}{\Theta}\alpha$

$\displaystyle \frac{I}{I_0}=B_{\infty}=\frac{2+\frac{1}{J}}{2}\coth \frac{2+\frac{1}{J}}{2}\alpha -\frac{1}{2J}\coth \frac{1}{2J}\alpha=\coth \alpha -\frac{1}{\alpha}$
\end{center}

いま、$J=\infty$なので$\displaystyle \frac{\alpha}{2J}<<1として第二式右辺第二項をマクローリン展開し、その第一項のみをとって計算した。$

\begin{center}
$\displaystyle すなわち、\frac{1}{2J}\coth \frac{\alpha}{2J}\simeq \frac{1}{2J}\frac{2J}{\alpha}=\frac{1}{\alpha}$とした。
\end{center}

上の2式のグラフの交点から$\alpha の値を求めて、式(20)に\alpha を代入してI/I_0-T/\Theta 曲線を求める。$

二式のグラフの交点から$\alpha$の値を求める方法を図9に示す。キュリー温度$\Theta$より高い温度では直線がブリルアン関数と原点以外で交点を持たないため考えないこととする。

\begin{figure}[h]
\centering
\includegraphics[width=6cm]{gs.jpg}
\caption{二式のグラフ的解法}
\end{figure}%

以下に(i)~(iii)の$I/I_0-T/\Theta 曲線を示す。$一つのグラフにまとめたため、グラフ中に凡例を示した。また、実験値のグラフも示した。

\begin{figure}[h]
\centering
\includegraphics[width=13cm]{k522.jpg}
\caption{$J=1/2,1,\infty,実験値のI/I_0-T/\Theta 曲線$}
\end{figure}%

$グラフから、J=1/2の場合が実験値ともっとも合う事が分かる。$

\newpage

\subsection{ジスプロシウム}

(1)ジスプロシウムはキュリー点($\Theta$~85K)の上に、磁気モーメントがらせん状に配列したらせん磁性相(85K$<T<$$\Theta_N$~180K)が存在する。キュリー点$\Theta およびニール点\Theta_N (らせん磁性相から常磁性相に転移する温度で、磁化率が極大値をとる。)の実験値を文献値と比較せよ。$

  

実験値と文献値を表2に示す。

\begin{table}[h]
  \centering
    \begin{tabular}{|r||r|r|} \hline
    
     & 実験値 & 文献値 \\ \hline
キュリー点$\Theta$ [K]& 92 & 90 \\ \hline
ニール点$\Theta_N$[K] & 181 & 179 \\ \hline
    
    \end{tabular}
    \label{tab:r_1}
\caption{キュリー点$\Theta$とニール点$\Theta_N$の実験値と文献値}
\end{table}

キュリー点は、課題(1)のニッケルのグラフで変化が最大の2点を近似直線で結び、その式を$y=0としてx軸の交点を求め、そこをキュリー点とした。$ニール点は磁化率が極大値をとる点すなわち、磁化率の逆数が極小値をとる点なので、課題(3)のグラフから実験値のニール点$\Theta_N$を求めた(図11)。

キュリー点もニール点も実験値が文献値より2K高い。キュリー点とニール点の温度が文献値よりも高い原因として考えられるのは、測定温度での磁化の大きさに減少しきる前に磁化の大きさを測定してしまったため、本来相転移する温度での磁化の大きさに達せず、より高い温度のときに相転移する磁化の大きさになったからだと考えられる。

実験値をより文献値に近づけるには、キュリー点やニール点付近の温度で、今回の実験よりもゆっくりと測定を行うという方法が考えられる。

\begin{figure}[h]
\centering
\includegraphics[width=6cm]{neel.jpg}
\caption{ニール点の場所}
\end{figure}%

(2)常磁性相($T>\Theta_N$)の磁化率の逆数の温度依存性がキュリーワイス則に従うかを確認せよ。また、その傾きからジスプロシウムの1原子当たりの有効磁気モーメント$\mu_B$を単位として実験的に求め、理論値($\mu_{eff}=10.63\mu_B$)と比較せよ。なお、傾きを$C$ [m/(H・K)]とすると、式(13)から$\mu_{eff}$の実験値は以下の式で与えられる。

\begin{center}
$\displaystyle \mu_{eff}=\frac{1}{\mu_B}\sqrt{\frac{3k}{NC}}$
\end{center}

ここで、$\mu_B=1.165\times 10^{-29}$ Wb・m、$k=1.3806\times 10^{-23}$ J/K、$N=3.17\times 10^{28}$m$^{-3}$

  

課題(3)のグラフをみると、ニール点以降で直線となっている。すなわち、磁化率の逆数が温度に比例しており、これはキュリーワイス則を満たしているということである。

  

課題(3)のグラフ中の近似直線の数式より、傾き$C=82533$ [m/(H・K)]である。なお、近似直線をとるうえで使った点は、ニール点付近ではない点で、点を結んだときに、なるべく直線になる点である。

以上の数字を与えられた式の代入すると、$\mu_{eff}=10.8 \mu_B$となる。

理論値と比較すると、値が大きいため今回の実験では傾き$C$[m/(H・K)]が小さかったということが分かる。傾きが小さかった原因として、測定した温度の磁化の大きさまで磁化が減少しきっていないときに測定してしまった(磁化率$\chi$が大きくなってしまう)ということが考えられる。

  

(3)ジスプロシウムの1原子あたりの有効磁気モーメントの理論値が$\mu_{eff}=10.63\mu_B$であたえられることを、パウリの排他律、フント則から説明しろ。

まず、$\mu_{eff}$は(14)式で与えられる。

\begin{center}
$\displaystyle \mu_{eff}=g_j\mu_B\sqrt{J(J+1)}$ (14)
\end{center}

$g_j$はランデの$g$因子といい、以下の式で表される。

\begin{equation}
\displaystyle g_j=\frac{3}{2}+\frac{S(S+1)-L(L+1)}{2J(J+1)}
\end{equation}

ここで、$Sはスピン角運動量、Lは軌道角運動量$。

ジスプロシウムの$S,L,Jを求める。$

電子の軌道は以下の四つの量子数で決められる。

\begin{center}
$主量子数 n:1,2,3,...$

$方位量子数 l:0,1,2,3,...,(n-1)$

$磁気量子数 m:0,\pm 1,\pm 2,...\pm l $

$スピン磁気量子数 s=-1/2,+1/2$
\end{center}

主量子数$n$はボーアの原子模型で言うK殻やM殻を指定する数で、方位量子数$l$は軌道s$(l=0)$,p$(l=1)$,d$(l=2),$f$(l=3)$...を指定する数であり、磁気量子数$m$は電子がその軌道に何個入ることができるかを示す数である。スピン磁気量子数$s$は電子のスピンの向きを表す数で1/2の時が上向きのスピンで-1/2の時が下向きのスピンである。電子が軌道に入るときの規則は以下のパウリの排他律とフントの規則に従う。

パウリの排他律とは、フェルミ粒子(電子など)は一つの軌道にはスピンの向きが互いに反対の電子2個までしか入れられないという法則である。

\newpage

フントの規則は以下の条件である。

\newtcolorbox{mysimplebox}[1]{%
 colframe=black, colback=white,
 coltitle=black, colbacktitle=white,
 boxrule=0.8pt, arc=0mm
 fonttitle=\sffamily\bfseries,
 enhanced,
 attach boxed title to top left={xshift=10mm,yshift=-3mm},
 boxed title style={frame hidden},
 title=#1}
 
\begin{mysimplebox}{フントの規則}
(1)電子のスピンはなるべく大きい全スピン$S(=\displaystyle \sum_i s_i)$を持つように平行になる。

(2)$S$最大の条件と矛盾しない範囲で全軌道角運動量$L(=\displaystyle \sum_i l_i)$が最大値をとる。

(3)全角運動量$J$は、電子殻が半分以下しか満たされていないときは$J=|L-S|、それ以上の時はJ=L+S$になる。
\end{mysimplebox}

Dy$^{3+}$の外殻は4f$^9$である。この時、方位量子数がf($l$=3)なので、磁気量子数$m_3$の取る値は7つある。パウリの排他律とフントの規則を考えて電子を軌道に入れていくと下図のようになる。電子は黒点で表している。ここで、$\uparrow$と
$\downarrow$はそれぞれスピンの上向きと下向きを表す。

\begin{figure}[h]
\centering
\includegraphics[width=5cm]{pf.jpg}
\caption{Dy$^{3+}$の基底状態}
\end{figure}%

このことから全スピン角運動量$S$を求めると、上向きのスピンが7個、下向きのスピンが2個であるためフントの規則(1)から

\begin{center}
$\displaystyle S=7\times \frac{1}{2}-2\times \frac{1}{2}=\frac{5}{2}$となる。
\end{center}

次に全軌道角運動量$L$を求めると、上向きのスピン電子は全体では軌道角運動量を持たないため下向きスピン電子2つだけの寄与が残るためフントの規則(2)から

\begin{center}
$\displaystyle L=3+2=5$となる。
\end{center}

最後に合成角運動量$J$を求める。殻が半分以上占有されているのでフントの規則(3)から、

\begin{center}
$\displaystyle J=5+\frac{5}{2}=\frac{15}{2}$となる。
\end{center}

こうしてジスプロシウムの$S,L,J$が求まったため、ランデの$g$因子の値を求める。(21)式に$S,L,J$の値を代入すると

\begin{center}
$\displaystyle g_j=\frac{3}{2}+\frac{\frac{5}{2}(\frac{5}{2}+1)-5(5+1)}{2\times \frac{15}{2}(\frac{15}{2}+1)}=\frac{4}{3}$
\end{center}

(14)式に$g_j,J$の値を代入すると

\begin{center}
$\displaystyle \mu_{eff}=\frac{4}{3}\sqrt{\frac{15}{2}(\frac{15}{2}+1)}\mu_B=10.63\mu_{B}$
\end{center}

となり、ジスプロシウムの1原子あたりの有効磁気モーメントの理論値が$\mu_{eff}=10.63\mu_B$であたえられることを、パウリの排他律、フント則から説明できた。



\newpage
\begin{thebibliography}{9}
\item 国立天文台編 「理科年表 2021」 丸善 2021年
  
\item 能勢宏、佐藤徹哉著 「磁気物性の基礎」 裳華房 1997年

\item 近角聰信著 「強磁性体の物理(上)」 裳華房 1988年

\item Stephen Blundell著 中村裕之訳 「固体の磁性」 内田老鶴圃 2015年

\item 作道恒太郎著 「固体物理-磁性・超伝導-」 裳華房 2004年

\item 小出昭一郎著 「量子力学(II)」 裳華房 2019年

\item 大野惇吉著 「新版 大学生の化学」 三共 2019年
  
\end{thebibliography}


































\end{document}