\documentclass[dvipdfmx]{jsarticle}
\usepackage[dvipdfmx]{graphicx}
\graphicspath{{ピクチャ/}}
\usepackage{otf}
\usepackage[dvipdfmx]{graphicx}
\usepackage{amsmath,amssymb}
\usepackage{url}
\begin{document}

\title{\huge II-6 超伝導体の評価}
\date{}
\maketitle

\begin{flushright}
\vspace{11cm}\Large
所属:物理・材料理工学科\\数理・物理コース\\
学籍番号:s0319007\\
氏名:上野智也\\
共同実験者:上野健斗\\片野峻太郎\\川瀬悠太\\
実験実施日:2021/11/11\\
レポート提出日:2021/11/12
\end{flushright}

\thispagestyle{empty}

\newpage

\section{目的}

高温超伝導体の電気抵抗の温度依存性を測定する。

\section{原理}

多くの単体の金属とその合金、さらに有機物や金属酸化物セラミックスなどで超伝導現象が見いだされている。これらの超伝導体ではある温度(臨界温度$T_c$以下)では、超伝導状態となる。その状態が示す特性として通常次のようなものが挙げられる。

  

(1)電気抵抗が零となる(抵抗0)。したがって、たとえばリング状の試料に一度電流が流れ始めるとずっと流れ続ける(永久電流)。

(2)完全反磁性を示す(マイスナー効果)。したがって、外部からの磁場があっても磁束は完全に外にはじき出される。

(3)フェルミ準位付近に電子のエネルギーのとびができる(エネルギーギャップ)。これにより、例えば臨界温度$T_c$で比熱に異常がみられる。

(4)外部からかける磁場がある値よりも大きくなると超伝導状態が壊れる(臨界磁場$H_c$)。

(5)流す電流がある値よりも大きくなると、超伝導状態が壊れる(臨界電流$I_c$)。

  

このように超電導体はただ単に電気抵抗が零となるだけではなく、多くの特異な性質を示す。そして、これらの特性は理学的にも、また工学的に関しても重要なものである。

従来では、超電導体の実験に低温技術上の種々の制約があり、困難が伴った。最近発見された高温超伝導では$T_c$が高く、
寒剤として液体窒素(沸点77K)を用いて行える手軽な実験が可能となった。ただし、高温超電導体は未だに開発の途上であり、従来の典型的な金属、合金超伝導体ほどその特性がはっきりと把握されていない。したがって、この実験では電気抵抗が確かに臨界温度$T_c$で急激に減少することをみる事にとどめる。これは、他の超伝導体の共通したもっとも基本的な性質でもある。

\section{使用機器}

試料(イットリウム系酸化物超電導体)、温度センサー(白金抵抗温度計)、液体窒素デュワー瓶、試料抵抗測定用定電流電源
(0~100mA)、白金抵抗測定用電流電源(1mA)、X-Yレコーダ

\section{実験}

(1)試料はYBa$_2$Cu$_3$O$_{7-y}$の組成を持つ酸化物超伝導体。この物質は複雑な結晶構造をもつが、通常の試料は単一の結晶体ではなく、焼結により作製された不均一な状態にある。組成式中でyの値が小さいほど$T_c$は高くなり(たとえば、y=0.07,0.40;$T_c\simeq 90,40$ K)、極端にyが大きいと(y$>$0.60)超電導体でなくなり、半導体的な挙動を示すようになる。ここで用いる試料は、$T_c=90$ Kのもので、大きさは5$\times 5 \times 10$ mm程度。試料の両端近くに2つの電流用端子、そのすぐ内側に2つの電圧用端子として細い導線がはんだ付けしてある(4端子抵抗測定法)。温度測定は白金の電気抵抗の温度変化を利用する(白金抵抗温度計)、温度計は試料のなるべく近くに取り付けてある。これらは試料ホルダーに収められ、電気リード線は外に取り出してある。

(2)試料に10 mAの一定の測定用電流を流す。試料の抵抗値は$T_c$以上で1~10 mΩであるから、出力電圧は10~300 μVになる。これをX-Yレコーダの縦軸に入れる。

(3)白金抵抗センサーに1 mAの電流を流す。その抵抗は0℃で100Ω程度なので、出力電圧はそこで100mVとなる。これをX-Yレコーダの横軸に入れる(図1参照)。

(4)デュワー瓶に液体窒素を入れ、試料容器をゆっくり降下し液中に浸す。X-Yレコーダ上の動きによく注意すること。ある温度で出力電圧が零(抵抗0)になるはずである。このときに試料の抵抗特性を記録しておく。

(5)記録された曲線から適当な数のデータを選び、白金センサーの抵抗-温度特性表を用いて横軸を温度に換算する。また、縦軸を試料の電気抵抗に換算する。これらのデータをプロットしてみれば、図2のような結果が得らえるはずである。(ここで、オンセット温度とは温度が降下していったときに初めて抵抗が急激に下がった時の温度、オフセット温度とは抵抗が0近くで一定になり始めた時の温度である。)

(6)測定用電流を30mAにしてもう一度実験を行う。

\begin{figure}[h]
\centering
\includegraphics[width=10cm]{1.jpg}
\caption{測定回路}
\end{figure}%

\newpage

\begin{figure}[h]
\centering
\includegraphics[width=8cm]{2.jpg}
\caption{超電導体の電気抵抗の温度依存性}
\end{figure}%

\section{注意事項}

(1)液体窒素で凍傷を起こさないように注意する。

(2)試料は熱ショックに弱いため、温度の降下、上昇を急に行わないこと。常にX-Yレコーダに注目すること。
\newpage

\section{結果}

以下の2つの表はそれぞれ10mA,30mAの時に記録された曲線から適当なデータ点を選び、xとyそれぞれの座標を温度(K)と抵抗(Ω)に変換した結果の表である。変換方法は表に書いてある通りに行った。

\begin{figure}[h]
\centering
\includegraphics[width=11cm]{10ma.pdf}
\caption{10mAの時のデータの変換}
\end{figure}%

\newpage

\begin{figure}[h]
\centering
\includegraphics[width=11cm]{30ma.pdf}
\caption{30mAの時のデータの変換}
\end{figure}%
\newpage

以上の2つの表から温度と抵抗の点をプロットすると以下の表が得られる。この結果から今回の目標である「高温超伝導体の電気抵抗の温度依存性を測定する」が達成できたといえる。また、オンセット温度とオフセット温度は課題(1)に記した。

\begin{figure}[h]
\centering
\includegraphics[width=12cm]{graph.pdf}
\caption{超電導体の電気抵抗の温度依存性}
\end{figure}%
\newpage


\section{課題}

(1)超電導状態に入り始める(オンセット)温度と、完全に超電導体となる(オフセット)温度をデータから求めろ。

・10mAの時
\begin{center}
オンセット温度:94.8794 K,オフセット温度:92.993 K
\end{center}

  

(2)試料に流す電流が30mAの時のオンセット、オフセット温度は変わるか

\begin{center}
オンセット温度:94.6436 K,オフセット温度:91.814 K
\end{center}

この値は10mAの時とほとんど変わらない事が分かる。

  

(3)マイスナー効果を観察する方法

超電導体を液体窒素などの寒剤で臨界温度まで冷やし、強力な磁石を近づけて反発の力を確認することによって観察できる。

  

(4)永久電流を実現するための実際の方法

・永久電流スイッチ

超伝導体でできたワイヤーを巻いて作ったコイルを超電導コイルという。このコイルを臨界温度以下にして電源から電流を流して磁場を得ることができる。しかし、電源を切ると電流が消えると一緒に磁場も消えてしまう。そこで、下図のように超電導コイルと並列になるように回路中にスイッチを作る。コイルの電流を増やしているときは、スイッチをオフにしておく。電流が望む大きさになったらスイッチをオンにする。コイルはスイッチでショートされるから電源を切ってもコイルには電流が流れ続ける。これを永久電流といい、このスイッチを永久電流スイッチという。

スイッチをオフにすると言っても機械的にオフにするわけではない。永久磁石スイッチは小さく無誘導に巻いたコイルである。コイルは、常伝導の時で数10Ω、超伝導の時で0Ωである。常伝導がオフ、超伝導がオンに対応している。オンとオフの切り替えはヒーターを用いて行う。オフへの切り替えを効率よく行うために、ヒーターの熱を逃がさない構造になっている。オフの時の抵抗を大きくするためにワイヤーの母材には銅ニッケル合金のような高抵抗率の材料を使う。

永久電流スイッチを用いている機械の代表例は理化学用NMRや医療用MRI。この装置では、良質なスペクトルや画像を得るには、減衰しない安定な磁場が必要。NMRで磁場の減衰率は1時間あたり1億分の1以下というとても小さな値だが、このような高い安定性は永久電流スイッチを用いた永久電流で初めて得られる。

\newpage


\begin{figure}[h]
\centering
\includegraphics[width=6cm]{device.jpg}
\caption{永久電流スイッチの回路図}
\end{figure}%

  

(5)超電導体のエネルギーギャップの存在によって、どのような物理現象が生じるか、調べる。

フォトンのエネルギー$\hbar \omega$が超伝導ギャップを超えた時のみ、電磁波の吸収が起こる。

超伝導状態での電子の比熱への寄与は-1/Tに比例する変数を持った指数関数の形である。これは、電子があるエネルギーギャップを超えて励起されることを示す。


\begin{thebibliography}{9}
\item 「超伝導web21 2002年1月号」 国際超伝導産業技術研究センター 2002年

\item H・イバッハ,H・リュート著 石井・木村訳 「固体物理学 新世紀物質科学への基礎」
\\シュプリンガー・フェアラーク東京 1998年
  
\item キッテル著 宇野ら訳 「第8版 キッテル固体物理学入門(上)」 丸善 2005年
  
\end{thebibliography}



















\end{document}