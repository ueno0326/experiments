\documentclass[dvipdfmx]{jsarticle}
\usepackage[dvipdfmx]{graphicx}
\graphicspath{{ピクチャ/}}
\usepackage{otf}
\usepackage[dvipdfmx]{graphicx}
\begin{document}

\title{\huge I-1 材料化学分析}
\date{}
\maketitle

\begin{flushright}
\vspace{11cm}\Large
学籍番号:s0319007\\
氏名:上野智也\\
共同実験者:荒巻俊太 
五十嵐武\\伊藤絵美里 伊藤優希\\伊藤龍平 上野健斗\\片野峻太郎 川瀬悠太\\水島悠人 小野寺晴海\\
実験実施日:2021/5/13\\
レポート提出日:2021/5/21
\end{flushright}

\thispagestyle{empty}
\newpage
\section{目的}

金属材料の品質の評価・管理において、その材料の成分と濃度を知ることは重要。本実験は、誘導結合プラズマ発光分光法(inductively coupled plasma -optical spectrometry,ICP-OES)によって銅合金中の主成分の濃度を分析する。

\section{原理}

原子にを与えて高いエネルギー準位に励起すると、ごく短い時間内に基底状態にに戻るがこの際、元素固有の光を放出する。この発光スペクトルを測定すると、波長から元素の種類が分かり、強度から原子の量が分かる。誘導結合プラズマ発光分光法はアルゴンプラズマ中に溶液試料を噴霧して原子を励起し、得られる発光スペクトルを測定することで、溶液試料中に含まれる元素の種類と濃度を測定する分析方法である。ICP-OESは以下のような特徴を持つ。

(1)測定する溶液は有機、無機関係なし

(2)多数の元素を同時測定できる。

(3)測定ダイナミックレンジ(測定試料中の目的元素の濃度とその発光強度が直線関係を示す範囲)が広い

(4)高感度で測定できる。(検出下限が数10~数$\mu$g/L)

(5)共存元素による化学干渉やイオン干渉が少ない。

これらの特徴からICP=OESは幅広い分野で用いられている。

ICP-OESでは誘導コイルに高周波電流を流すことによってトーチ内を流れるアルゴンガスをプラズマ化し、これに試料を導入する。試料ようえきをペリスタルティックポンプで吸収し、ネブライザーを使ってキャリアガスと共にチャンバー内に噴霧する。チャンバー内では霧化された試料溶液の粒径が選別され、細かいものがアルゴンプラズマ内に導入される。試料中の各原子はアルゴンプラズマ中で発光する。本実験では、プラズマの側面から発光スペクトルを測定(ラジアル測光)する。

ICP-OESは溶液を分析する方法なので、金属材料などの個体を分析するときは、金属材料を酸等に分解し、適宜希釈して試料溶液とする必要がある。測定元素の定量分析は、各元素濃度が既知の溶液の発光スペクトルから検量線を作成し、これを用いて行う。試料溶液中の各元素濃度が検量作成用標準液の濃度範囲になるように適宜希釈して測定。

ICP-OESでは溶液中に含まれる元素の種類や量によって、様々な形のスペクトルが得られる。初めに、測定された発光スペクトルを見て、各元素濃度の定量に用いる分析波長を選択する。分析波長は原則的に発光強度が充分に高いものを選択する。共存元素によっては図1のように分光干渉を受けている事があるので、このような波長を用いるのはできるだけ避けるべきである。

つぎに、各標準液および試料溶液中の測定元素の発光強度を求める。ここでは、分光波長における発光強度からバックグラウンドを差し引いたものを測定元素の発光強度とする。バックグラウンドには測定波長前後で一様に強度が上昇するものや、波長依存性を示すものがある。標準液と試料溶液間に粘性や密度などの物理的性質の差がある場合にこのようなバックグラウンドが生じる可能性がある。
\newpage
図2にバックグラウンド強度の評価方法の一例を示す。図2(a)のように、バックグラウンド強度$I_b$が測定波長前後で一様な場合には測定波長における発光強度$I_0$から引くだけでよい。

\begin{equation}
I=I_0-I_b
\end{equation}

図2(b)のようにバックグラウンド強度$I_b$が測定波長に比例するような場合、測定波長を通り、ピークのすそに接するような直線をひいて、これをバックグラウンドとする。その直線上で、測定波長から左右に距離の等しい点の強度を$I_{b1},I_{b2}$と置けば、次の式のように求められる。

\begin{equation}
I=I_0-I_b=I_0-(I_{b1}-I_{b2})/2
\end{equation}

\begin{figure}[h]
\centering
\includegraphics[width=8cm]{分光干渉.jpg}
\caption{二種類の元素による分光干渉の例}
\end{figure}%

\begin{figure}[h]
\centering
\includegraphics[width=8cm]{バックグラウンド.jpg}
\caption{バックグラウンド処理の例}
\end{figure}%

\newpage
\section{実験方法}

{\large 用いる試料、試薬、実験器具}

試料:銅合金

試薬:銅標準液(1000mg/L)、ニッケル標準液(1000mg/L)、亜鉛標準液(1000mg/L)、(1+1)硝酸(濃硝酸原液を蒸留水で容量二倍に希釈したもの)、蒸留水

実験器具:コニカルビーカー(300mL)、グリフィンビーカー(100mL,1000mL)、ホールピペット(5mL,10mL)、メスフラスコ(100mL)

  

{\large 検量線作成用標準液の作成}

1.100mLグリフィンビーカー、10mLホールピペットを少量の各標準溶液(1000mg/L)で共洗いし、ビーカーに標準液(1000mg/L)を20mL程度入れる。

2.ビーカーから標準液(1000mg/L)を10mLずつ取り、100mLメスフラスコにすべて入れる。その後(1+1)硝酸を10mL加え、蒸留水を表戦まで入れて検量線作成用標準液(100mg/L)を作成

3.検量線作成用標準液(100mg/L)を共洗いしたホールピペットで10mL分取し、100mLメスフラスコに入れ、(1+1)硝酸を10mL加え、
蒸留水を標線まで入れて10倍に希釈し、検量線作成用標準液(10mg/L)を作成。

4.3と同様の手順で検量線作成用標準液(1mg/L)を作成。また、5mLホールピペットを用いて検量線作成用標準液(5mg/L)を作成。

5.100mLメスフラスコに(1+1)硝酸を10mL加え、蒸留水を標線まで入れてブランク溶液(0mg/L)を作成。

  

{\large 資料溶液の作成}

1.銅合金試料を0.2g取り、電子天秤で重量を精度よく測定

2.銅合金をコニカルビーカーに入れ、(1+1)硝酸10mL程度加えて、ホットプレートに置き、加熱溶解する

3.残渣がないことを目視で確認し、室温まで冷却し100mLメスフラスコに注ぎ入れ、蒸留水を標線まで加える。

4.3で製作した溶液を共洗いしたホールピペットで10mL分取し、(1+1)硝酸10mLを加えたのち、100mLメスフラスコに注ぎ入れ、蒸留水を標線まで加えて10倍に希釈する。

5.4を繰り返し、3で作成した溶液を100倍および1000倍に希釈した試料溶液を作製する。

  

{\large 原子発光スペクトルの測定}

1.IPC-OES装置を立ち上げる。

2.検量線作成用標準液、試料溶液の発光スペクトルを測定する。

3.測定結果を印刷する。

\newpage
\section{結果}

1.測定された発光スペクトルの強度を表にまとめる。

\begin{table}[h]
  \centering
    \caption{各溶液中のCu,NI,Znの発光強度}
    \begin{tabular}{|l|r|r|r|} \hline
    測定元素 & Cu & Ni & Zn \\ \hline
測定波長(nm) & 224.700 & 221.647 & 206.200 \\ \hline
ブランク(0mg/L) & 0 & 4 & 3 \\ \hline
標準液(1mg/L) & 263 & 617 & 1188 \\ \hline
標準液(5mg/L) & 1313 & 3077 & 5941 \\ \hline
標準液(10mg/L) & 2605 & 6121 & 11888 \\ \hline
試料B(1000倍希釈) & 458 & 125 & 6 \\ \hline
試料B(100倍希釈) & 4486 & 1192 & 16 \\ \hline
試料B(10倍希釈) & 43205 & 11507 & 138 \\ \hline
試料C(1000倍希釈) & 287 & 143 & 842 \\ \hline
試料C(100倍希釈) & 2758 & 1379 & 8310 \\ \hline
試料C(10倍希釈) & 26597 & 13362 & 80360 \\ \hline
    
    \end{tabular}
    \label{tab:r_1}
\end{table}

2.標準液の濃度[A](A=Cu$^{2+}$,Ni$^{2+}$,Pb$^{2+}$)[mg/L]をx軸、測定波長における発光スペクトルの強度I[a.u.]をy軸として、銅、ニッケル、亜鉛の検量線をそれぞれ作成し、近似曲線の式を求める。それぞれのグラフの点はそれぞれの元素の測定波長におけるスペクトルの強度で、破線は近似直線である。下のグラフは銅の検量線である。

\begin{figure}[h]
\centering
\includegraphics[width=12cm]{銅検量線.png}
\caption{銅の検量線}
\end{figure}%
\newpage

下のグラフはニッケルの検量線

\begin{figure}[h]
\centering
\includegraphics[width=12cm]{ニッケル検量線.png}
\caption{ニッケルの検量線}
\end{figure}%

下のグラフは鉛の検量線

\begin{figure}[h]
\centering
\includegraphics[width=12cm]{亜鉛検量線.png}
\caption{亜鉛の検量線}
\end{figure}%
\newpage

それぞれの近似直線の式をまとめると($\pm$以下の数字はExcelを用いて求めた傾きの標準誤差)

\begin{table}[h]
  \centering
    \caption{それぞれの検量線の近似直線の式}
    \begin{tabular}{|r|r|} \hline
    & 近似直線の式 \\ \hline
Cu & $(261\pm 0.773)[Cu^{2+}]$ \\ \hline
Ni & $(613\pm 1.21)[Ni^{2+}]$ \\ \hline
Zn & $(1189\pm 0.220)[Zn^{2+}]$ \\ \hline
    \end{tabular}
    \label{tab:r_1}
\end{table}

3.以下の式を用いて銅合金試料中の銅、ニッケル、亜鉛の濃度を求める。

\begin{equation}
合金中の成分濃度[mass\%]=\frac{I/(a\pm \Delta a)\times 10^{-3} \times 希釈率\times 0.1}{試料重量[g]} \times 100
\end{equation}

計算方法として$\displaystyle \frac{I\times 10^{-3}\times 希釈率\times 0.1\times100}{試料重量[g]}$をはじめに計算し、最後に$(a\pm \Delta a)で割ることにした。$

下表が合金中の成分濃度の表だが、銅合金Bの亜鉛の発光強度は亜鉛の標準液(1mg/L)の値より小さかったため計算しなかった。

\begin{table}[h]
  \centering
    \caption{銅合金中の成分濃度}
    \begin{tabular}{|r|r|r|} \hline
    合金中の成分濃度 & 銅合金B & 銅合金C \\ \hline
Cu[mass\%] & 88.1$\pm 0.261$ & 54.0$\pm 0.160$ \\ \hline
Ni[mass\%] & 9.76$\pm 0.0193$ & 11.1$\pm 0.0219$ \\ \hline
Zn[mass\%] &  & 34.53$\pm 0.006340$ \\ \hline
    \end{tabular}
    \label{tab:r_1}
\end{table}

\section{考察}

銅合金Bの組成は銅約88.1\% ニッケル約9.76\% でその和をとると約97.9\%となる。Znの発光強度の値が小さすぎて割合が出せなかったが、多く入っていても約2\% と考えられる。

合金Cの組成は銅約54.0\% ニッケル約11.1\% 鉛約34.53\% でその和をとると約99.6\%となる。残りの0.4\% は不純物だと考えられる。

すべてのグラフの点が近似直線上にあるため、1000倍以上の希釈しても近似直線上に点が打てると予想でき、このことから、発光強度が元素濃度に比例していることが改めて確認できる。

また、標準液の濃度はすべて同じなので、発光強度の値から、Zn>Ni>Cuの順に発光強度が大きいことが分かる。


\newpage
\section{レポート課題}

1.銅-ニッケル-亜鉛合金の種類と用途を調べて書く

  

(1)洋銀

金属の割合は10~20\% Ni,15~30\% Zn, 50~70\% Cu

Niは洋銀の必要成分で、銀白色、耐銹性、硬度を与える。

銀の代用品、電気抵抗線として使われている。電気抵抗は主としてNiの含有量で決まる。Ni 7~30\% の範囲では大体Ni\% +12
(microhm cm)で与えられる。

  

(2)マンツメタル(6:4真鍮)

金属の割合は60\% Cu, 40\% Zn

高力なため、機械部品として幅広く用いられている。海水中で均一に腐食を受け美観を呈するので船体の外装板等にも用いられる。

延伸率が小さい

  

(3)シルジン青銅

金属の割合は79~83\% Cu, 10~16\% Zn,5\% Si

バルブ、コック、パッキングリング、管接手等に用いられる。

耐海水腐食がある。

  

2.ICP-OESにおける内標準法について調べて書く

  

内標準法とは、x軸に測定元素の濃度、y軸に測定元素と都内標準元素の発光強度をとり、検量線を作成し、サンプルを測定した時の測定元素とない標準元素との発光強度比から濃度を算出する手法である。内標準法は物理的干渉を補正するのに適した分析方法である。しかし、内標準元素の選択を誤ると定量値が大きく異なる可能性があるので注意が必要。内標準元素として適しているのは発光強度が大きい、分光干渉がない、サンプル中に含まれない、測定元素と分光特性が似ているというものである。分光特性が似ているというのは線種を可能な限りそろえるということ。(例:原子線を測定するなら原子線のない標準元素を選ぶ)さらに、測定元素と内標準元素の励起エネルギーの差が小さい事を意味する。内標準元素を適切に選択すると、イオン干渉もある程度補正可能になる。

\begin{thebibliography}{9}
\item 濱住松二郎 「非鉄金属及び合金」 内田老亀圃新社 1972年

\item アジレント・テクノロジー株式会社 「ICP発光分光分析装置 (ICP-OES) の基礎」\\
(URL) https://www.chem-agilent.com/contents.php?id=1001752  閲覧日2021/5/14
  
\end{thebibliography}












\end{document}